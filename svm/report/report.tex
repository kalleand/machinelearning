\documentclass[a4paper]{article}

%\usepackage[cm]{fullpage}
\usepackage[T1]{fontenc}
\usepackage[english]{babel}
\usepackage{textcomp}
\usepackage{multirow}
\usepackage{float}
\usepackage{fancyhdr}
\usepackage{pdfpages}

\usepackage{graphicx}
\usepackage{amsmath}

\title{Support Vector Machine: Lab 2}
\author{Karl Johan Andreasson <\href{mailto:kalleand@kth.se}{kalleand@kth.se}> %
\and Christian Wemstad <\href{mailto:wemstad@kth.se}{wemstad@kth.se}> %
}

\fancyhf{}
\fancyhead[LE,RO]{\slshape \rightmark}
\fancyhead[LO,RE]{\slshape \leftmark}
\fancyfoot[C]{\thepage}

\begin{document}
\thispagestyle{empty}
\maketitle
\thispagestyle{empty}
\pagestyle{empty}
\newpage
\tableofcontents
\newpage
\pagestyle{fancy}
\setcounter{page}{1}
\section{Why G is -1 * I}
In the constraint of equation (12) $G\vec{\alpha}\geq \vec{h}$ this
gives us that  $-1*I*\vec{\alpha} \geq \vec{h} $. Which implies that $\alpha$ needs to be
greater than 0, since h is a vector with zeros.
\section{Kernels}
\subsection{Linear Kernel}
This is the most simpel Kernel function, it draws a linear line between the
datagroups to separate the datapoints. This kernel will fail if no linear
solution exists and slack is not allowed. 

\subsection{Polynomical Kernel}
Polynomical is the first non-linear function. It adds a varable $p$
to the equation that acts as a exponent to the linear kernel. We found that
$p=3$ is the best power for our dataset. It solves the issue without generating
too complex solutions.

\subsection{Radial Basis Function Kernel}
Radial Basis Function is another non-linear kernal. It has the varialbe
$\sigma$ as a global parameter for tuning the algroithm. We used $10e-5$ for
$\sigma$ to get a good solution to our dataset.

\subsection{Signoid Kernel}
This kernel uses $k$ and $\delta$ as global parameters to tune the algorithm. In
our test, for our data, we found that $k$ should be around 0.05 (1/N) and $\delta$
should be about 0.1
\end{document}
